%Chapter 7

\renewcommand{\thechapter}{7}

\chapter{Conclusion}
In conclusion, the results of this project will add valuable knowledge to the field of wireless power transfer 
by creating a mathematical model to predict power output of a DWPT system based on specific variables. 
Physics predicts that DWPT infrastructure modeled like that in our simulation would be able to transfer enough 
power to increase the range of electric vehicles significantly. Furthermore, the project will demonstrate the 
accuracy of the model by applying it to a real-world small-scale experimental model.  If successful, this model 
can be used in the design of a DWPT system to be implemented in existing highways after more testing. This project 
has also shown that a valuable charging rate could be established for 875 \$/C.  This is only a relative cost, 
so further research should be done into cost models of these systems. 

There are multiple extensions of our project that can be pursued by other research teams building upon the 
limitations discussed here. Our project sought to simulate the use of a dynamic wireless charging system in o
rder to recommend optimal operating conditions and to generate a preliminary cost analysis. Future projects could 
include a policy recommendation report based on the findings of our research, which would discuss feasibility and 
cost of implementation in depth. There could be another project solely dedicated to modifying our simulation to 
better reflect real world conditions, taking changes in weather and material properties into consideration. We 
would also recommend a project focused on improving our physical experimental model and comparing any data collected 
to the results of the simulation discussed in this thesis. Our research is just one step in working towards 
encouraging the adoption of electric vehicles and reducing harmful vehicle emissions. There are many more avenues 
to be taken in continuing to build the gas station of the future.