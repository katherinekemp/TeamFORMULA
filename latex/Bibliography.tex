\begin{thebibliography}{99}
\setlength{\parskip}{1em}

\bibitem{Agrawal1} G.P. Agrawal, {\em Nonlinear Fiber Optics} (Academic
Press, San Diego, CA, 2001), Chap. 1.
\bibitem{“2017 Hybrid/EV Survey Results.”} CarMax (2017), CarMax, www.carmax.com/articles/hybrid-electric-2017-survey-results
\bibitem{“2021 Nissan LEAF Range, Charging & Battery.”} (n.d.). Retrieved March 20, 2021, from https://www.nissanusa.com/vehicles/electric-cars/leaf/features/range-charging-battery.html
\bibitem{Arendash}, G. W., Sanchez-Ramos, J., Mori, T., Mamcarz, M., Lin, X., Runfeldt, M., … Cao, C. (2010). Electromagnetic Field Treatment Protects Against and Reverses Cognitive Impairment in Alzheimer’s Disease Mice. Journal of Alzheimer’s Disease, 19(1), 191–210. https://doi.org/10.3233/JAD-2010-1228
\bibitem{Arribas-Ibar} M, Nylund PA, Brem A. The Risk of Dissolution of Sustainable Innovation Ecosystems in Times of Crisis: The Electric Vehicle during the COVID-19 Pandemic. Sustainability. 2021; 13(3):1319. https://doi.org/10.3390/su13031319
\bibitem{Banerjee}, Debolina, et al. (2018) “I-1631 Takes Aim at the True Cost of Fossil Fuel Use for Communities of Color.” Puget Sound Sage, www.pugetsoundsage.org/true-cost-of-fossil-fuel-use-for-communities-of-color/
%\bibitem{Chen}, Z., Xu, C., Ma, C., Ren, W., & Cheng, H.-M. (2013). Lightweight and Flexible Graphene Foam Composites for High-Performance Electromagnetic Interference Shielding. Advanced Materials, 25(9), 1296–1300. https://doi.org/10.1002/adma.201204196
%\bibitem{Clement-Nyns}, K., Haesen, E., & Driesen, J. (2010). The Impact of Charging Plug-In Hybrid Electric Vehicles on a Residential Distribution Grid. IEEE Transactions on Power Systems, 25(1), 371–380.
%\bibitem{Eberhard}, M., & Tarpenning, M. (2018). The 21 st Century Electric Car Tesla Motors.
%\bibitem{Fisher}, T. M., Farley, K. B., Gao, Y., Bai, H., & Tse, Z. T. H. (2014). Electric vehicle wireless charging technology: a state-of-the-art review of magnetic coupling systems. Wireless Power Transfer, 1(02), 87–96. https://doi.org/10.1017/wpt.2014.8
\bibitem{Fujita}, T., Yasuda, T. and Akagi, H. (2017, July-August). A Dynamic Wireless Power Transfer System Applicable to a Stationary System. IEEE Transactions on Industry Applications, vol. 53, no. 4, pp. 3748-3757, doi: 10.1109/TIA.2017.2680400.
%\bibitem{Helmers}, E., & Marx, P. (2012). Electric cars: Technical characteristics and environmental impacts. Environmental Sciences Europe, 24(1). doi:10.1186/2190-4715-24-14
\bibitem{Houglum}, M. (2015). Oauth2client. https://github.com/googleapis/oauth2client 
%\bibitem{Ho}, S. L., Wang, J., Fu, W. N., & Sun, M. (2011). A Comparative Study Between Novel Witricity and Traditional Inductive Magnetic Coupling in Wireless Charging. IEEE Transactions on Magnetics, 47(5), 1522–1525. https://doi.org/10.1109/TMAG.2010.2091495
%\bibitem{Hunter}, J. D. (2007), Matplotlib: A 2D graphics environment, In 2007 Computing in Science & Engineering (pp. 90-95).
\bibitem{Joselow}, Maxine (2021) “Gasoline Car Sales to End by 2035 in Massachusetts” Scientific American. 
%\bibitem{J2954}: Wireless Power Transfer for Light-Duty Plug-in/Electric Vehicles and Alignment Methodology - SAE International. (n.d.). Retrieved April 1, 2021, from https://www.sae.org/standards/content/j2954_202010/
%\bibitem{Karam Hwang}, Jaeyong Cho, Dongwook Kim, Jaehyoung Park, Jong Hwa Kwon, Sang Il Kwak, … Seungyoung Ahn. (2017). An Autonomous Coil Alignment System for the DynamicWireless Charging of Electric Vehicles to Minimize Lateral Misalignment. Energies (19961073), 10(3), 1–20. https://doi.org/10.3390/en10030315
%\bibitem{Kostoff}, R. N., & Lau, C. G. Y. (2013). Combined biological and health effects of electromagnetic fields and other agents in the published literature. Technological Forecasting and Social Change, 80(7), 1331–1349. https://doi.org/10.1016/j.techfore.2012.12.006
\bibitem{Li, Y., Wang, W., Xing, L., Fan, Q., & Wang, H. (2018). Longitudinal safety evaluation of electric vehicles with the partial wireless charging lane on freeways. Accident Analysis & Prevention, 111, 133–141. https://doi.org/10.1016/j.aap.2017.11.036
\bibitem{Lu}, X., Wang, P., Niyato, D., Kim, D. I., & Han, Z. (2016). Wireless Charging Technologies: Fundamentals, Standards, and Network Applications. IEEE Communications Surveys & Tutorials, 18(2), 1413–1452. https://doi.org/10.1109/COMST.2015.2499783
\bibitem{Lukic}, S., & Pantic, Z. (2013). Cutting the Cord: Static and Dynamic Inductive Wireless Charging of Electric Vehicles. IEEE Electrification Magazine, 1(1), 57–64. https://doi.org/10.1109/MELE.2013.2273228
\bibitem{Lumpkins}, W. (2014). Nikola Tesla’s Dream Realized: Wireless power energy harvesting. IEEE Consumer Electronics Magazine, 3(1), 39–42. https://doi.org/10.1109/MCE.2013.2284940
\bibitem{Model x.} (n.d.). Retrieved April 02, 2021, from https://www.tesla.com/modelx
\bibitem{Mufson}, Steven “General Motors to eliminate gasoline and diesel light-duty cars and SUVs by 2035” The Washington Post. 28 Jan 2021.
\bibitem{Nabel}, R pydrive (2016) Retrieved March 30, 2021 from
https://pypi.org/project/PyDrive/ 
\bibitem{Nhede}, Nicholas (2020). “New Study Reveals Interesting Statistics on EV Ownership and Consumer Interest.” Smart Energy International, 16 Jan. 2020, www.smart-energy.com/industry-sectors/smart-energy/new-study-reveals-interesting-statistics-on-ev-ownership-and-consumer-interest/
\bibitem{Ogneva-Himmelberger}, Yelena, and Liyao Huang (2015). “Spatial Distribution of Unconventional Gas Wells and Human Populations in the Marcellus Shale in the United States: Vulnerability Analysis.” Applied Geography, vol. 60, June 2015, pp. 165–174., doi:10.1016/j.apgeog.2015.03.011.
\bibitem{Onar}, O. C., Miller, J. M., Campbell, S. L., Coomer, C., White, C. P., & Seiber, L. E. (2013). A novel wireless power transfer for in-motion EV/PHEV charging. In 2013 Twenty-Eighth Annual IEEE Applied Power Electronics Conference and Exposition (APEC) (pp. 3073–3080). Long Beach, CA, USA: IEEE. https://doi.org/10.1109/APEC.2013.6520738
\bibitem{Onat}, N. C., Noori, M., Kucukvar, M., Zhao, Y., Tatari, O., & Chester, M. (2017). Exploring the suitability of electric vehicles in the United States. Energy, 121, 631–642. https://doi.org/10.1016/j.energy.2017.01.035
\bibitem{Panchal}, C., Stegen, S., & Lu, J. (2018). Review of static and dynamic wireless electric vehicle charging system. Engineering Science and Technology, an International Journal, 21(5), 922–937. https://doi.org/10.1016/j.jestch.2018.06.015
\bibitem{Parmesh}, K., Neriya, R. P., & Kumar, M. V. (2017). Wireless Charging System for Electric Vehicles. International Journal of Vehicle Structures & Systems (IJVSS), 9(1), 23–26. https://doi.org/10.4273/ijvss.9.1.05
\bibitem{Pearre}, N. S., Kempton, W., Guensler, R. L., & Elango, V. V. (2011). Electric vehicles: How much range is required for a day’s driving? Transportation Research Part C: Emerging Technologies, 19(6), 1171–1184. https://doi.org/10.1016/j.trc.2010.12.010
\bibitem{Queval Loic} (2021). Biot Savart magnetic Toolbox (https://github.com/lqueval/BSmag), GitHub. Retrieved March 23, 2021.
\bibitem{Rakhymbay}, A., Khamitov, A., Bagheri, M., Alimkhanuly, B., Lu, M., Phung, T., … Phung, T. (2018). Precise Analysis on Mutual Inductance Variation in Dynamic Wireless Charging of Electric Vehicle. Energies, 11(3), 624. https://doi.org/10.3390/en11030624
\bibitem{Reback}, J., MacKinney, W., Jbrockmendal, Bossche, J.V.D., Augspurger, T., Cloud, P. (2021) pandas-dev/pandas: Pandas 1.2.3 (Version v1.2.3). Zenodo 
http://doi.org/10.5281/zenodo.4572994 
\bibitem{Resonant circuits} | Brilliant Math & Science Wiki. (n.d.). Retrieved December 11, 2018, from https://brilliant.org/wiki/resonant-circuits/
\bibitem{Skvarenina}, T. L. (2001). The Power Electronics Handbook. CRC Press.
\bibitem{Throngnumchai}, K., Hanamura, A., Naruse, Y., & Takeda, K. (2013). Design and evaluation of a wireless power transfer system with road embedded transmitter coils for dynamic charging of electric vehicles. In 2013 World Electric Vehicle Symposium and Exhibition (EVS27) (pp. 1–10). https://doi.org/10.1109/EVS.2013.6914937
\bibitem{Tiikkaja}, M., Aro, A. L., Alanko, T., Lindholm, H., Sistonen, H., Hartikainen, J. E. K., … Hietanen, M. (2013). Electromagnetic interference with cardiac pacemakers and implantable cardioverter-defibrillators from low-frequency electromagnetic fields in vivo. EP Europace, 15(3), 388–394. https://doi.org/10.1093/europace/eus345
\bibitem{US EPA, OA.} (2015). Sources of Greenhouse Gas Emissions [Overviews and Factsheets]. Retrieved October 1, 2018, from https://www.epa.gov/ghgemissions/sources-greenhouse-gas-emissions
\bibitem{US EPA, OAR.} (2016). Greenhouse Gas Emissions from a Typical Passenger Vehicle [Overviews and Factsheets]. Retrieved October 1, 2018, from https://www.epa.gov/greenvehicles/greenhouse-gas-emissions-typical-passenger-vehicle
\bibitem{Vilas}, M. google (2016) Retrieved March 30, 2021 from https://breakingcode.wordpress.com/
\bibitem{What is magnetic flux?} (n.d.). Retrieved October 1, 2018, from https://www.khanacademy.org/science/physics/magnetic-forces-and-magnetic-fields/magnetic-flux-faradays-law/a/what-is-magnetic-flux
\bibitem{Xiang}, L., Sun, Y., Tang, C., Dai, X., & Jiang, C. (2017). Design of crossed DD coil for dynamic wireless charging of electric vehicles. In 2017 IEEE PELS Workshop on Emerging Technologies: Wireless Power Transfer (WoW) (pp. 1–5). https://doi.org/10.1109/WoW.2017.7959422
\bibitem{Young}, H. D., & Freedman, R. A. (2016). University Physics with Modern Physics (14th ed.). Pearson.
\end{thebibliography}
