%Abstract Page

\hbox{\ }

\renewcommand{\baselinestretch}{1}
\small \normalsize

\begin{center}
\large{{ABSTRACT}}

\vspace{3em}

\end{center}
\hspace{-.15in}
\begin{tabular}{ll}
Title of Thesis:    & {\large  DYNAMIC WIRELESS POWER }\\
&                     {\large  TRANSFER USING DC POWER} \\
\ \\
&                          {\large  Team FORMULA} \\
&                           {\large Gemstone Honors College, 2021} \\
\ \\
Thesis Directed by: & {\large  Bryan Quinn} \\
&               {\large  Department of Electrical } \\
&               {\large  and Computer Engineering } \\
\end{tabular}

\vspace{3em}

\renewcommand{\baselinestretch}{2}
\large \normalsize
Constant stops for charging and lengthy recharging times make electric vehicles (EVs) inconvenient to operate for extended travel. Innovative charging methods are necessary if EVs are expected to gain traction in the market over the coming years. Current advancements allow EVs to be charged wirelessly while parked over a charging source. This method does not mitigate the issue of interrupting a trip to spend a significant amount of time charging the vehicle. We theorized that – by expanding on the current technology – EVs could be charged while in motion. The primary goal of this project was to develop a model that optimized the operation of a dynamic wireless power transfer (DWPT) system using DC power. Through a combination of digital simulations and physical tests, the team determined the factors that significantly impacted the power transfer to a receiving wire coil as it moved over a series of stationary transmitting coils. The results were used to confirm the feasibility of a DWPT system and to make recommendations as to the optimum operating conditions. 