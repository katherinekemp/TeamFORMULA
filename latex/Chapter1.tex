%Chapter 1

\renewcommand{\thechapter}{1}

\chapter{Introduction}

Sustainability is a critical point of focus within the United States’ transportation sector, 
yet the technology required to pave the way for vehicles that emit less greenhouse gases to 
take over has not been fully incorporated into the nation’s infrastructure \cite{onat_exploring_2017}.  
According to data collected in 2014 by Oak Ridge National Laboratory, 95\% of the transportation 
sector was reliant on fossil fuels in that year \cite{onat_exploring_2017}.  Additionally, in 2016, 
the U.S. Environmental Protection Agency estimated that 28\% of total greenhouse gas emissions 
in the United States were produced by the transportation sector and that the average passenger 
vehicle emitted nearly 4.6 metric tons of carbon dioxide gas into the atmosphere each year 
\cite{us_epa_greenhouse_2016}. These statistics highlight the adverse environmental effects that will result 
from the prolonged use of internal combustion engine (ICE) vehicles.

The disruption to the automotive industry caused by the COVID-19 pandemic has served as an 
opportunity for the innovative technology in electric vehicles to break through to the general 
public, as there have been widespread reductions in mandatory commuting \cite{arribas-ibar_risk_2021}. 
Environmental concern and awareness among consumers, sparked by the looming threat of climate 
change, has prompted major car companies, including General Motors and Jaguar, to shift their 
light-duty vehicle focus to exclusively EVs \cite{mufson_general_nodate}, and states like California and 
Massachusetts have announced plans to ban the sale of gasoline-powered cars within the next 
15 years \cite{news_gasoline_nodate}.

Despite this, the overwhelming majority of vehicles on the road continue to be powered by an ICE. 
EVs are the future of the automotive industry, but technological restrictions remain that must be 
addressed before they become viable replacements for ICE vehicles in the general market \cite{lukic_cutting_2013}. 
Two notable shortcomings of current EVs are their relatively low ranges and long charging times when compared 
to their ICE vehicle counterparts \cite{parmesh_wireless_2017}. As a result, long distance trips in EVs are too 
often interrupted by lengthy charging stops in between driving. This is unappealing to consumers still driving 
ICE vehicles, who frequently cite “range anxiety” as the main barrier to purchasing EVs 
\cite{throngnumchai_design_2013}. Though some consumers have already made the switch over to EVs, widespread adoption 
will not occur until this issue is addressed.

Numerous research efforts have already been launched in attempt to solve the problem of limited EV range 
\cite{panchal_review_2018}. Much of the existing research can be categorized into two main areas: optimizing 
the characteristics of the battery, such as the chemistry, capacity, size, etc., and external ways to 
extend the battery range \cite{lukic_cutting_2013}. Since current research into battery optimization has 
already produced what appears to be the most efficient battery based on size, we have chosen to look into 
the latter \cite{lukic_cutting_2013}.  Current research into external solutions, including solar panel 
integration, wireless charging, and charging with induced currents, are not fully developed, leaving 
room for exploration \cite{parmesh_wireless_2017}. Among these, designing a system utilizing induced currents 
to charge cars while in motion is what our team feels to be the most hopeful topic for extended research.

Our research focuses on design optimization and small-scale, practical installment of an in-road dynamic 
charging system. With the simulated tests run for our research, we look to fill in the existing gaps in 
knowledge pertaining to the optimal frequency, distance between transmitting and receiving coils, vehicle 
speed, and methods of energy creation within such a system. To do this, we will address the following 
research questions: 
\begin{enumerate}
    \item[(1)]
    How can the dynamic charging of EVs be successfully implemented into a roadway?  
    \item[(2)]
    Would dynamic charging be able to provide enough power to a vehicle to make a difference in its overall range capability? 
    \item[(3)]
    Is dynamic charging a safe and reliable means of transmitting power to a vehicle in motion on the roadway, and would it affect a vehicle’s performance in other ways?   
    \item[(4)]
    Would a traditional roadway be able to accommodate dynamic charging, or would more nontraditional infrastructure need to be created?
\end{enumerate}