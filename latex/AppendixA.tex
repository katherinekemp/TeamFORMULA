%Appendix -- January 2015
\appendix
\renewcommand{\thechapter}{A}
\renewcommand{\chaptername}{Appendix}

\chapter{Equity Impact Report}

The Oxford English Dictionary defines equity as “the quality of being fair and impartial.” 
The phrase “equitable research” implies that a study is being carried out in such a way that 
its results are inclusive, causing little to no harm to any one community. 
Team FORMULA is committed to benign and beneficial involvement in research. 
The purpose of this report is to examine the unintended consequences of our research so 
that we may formulate an equitable recommendation as to the use of our results.

The results of our research could be used in the future implementation of wire- less charging lanes. 
There are many factors to be considered for this infrastructure, such as installation costs, location, 
and revenue scheme. State and federal gas taxes typically fund highway expansion and maintenance. 
An increase in the use of elec- tric vehicles, encouraged by new charging infrastructure, could 
cause a decrease in the already dwindling revenue collected through these taxes. To mitigate this lost revenue, 
tolls could be applied to wireless charging lanes. A survey conducted by CarMax and CleanTechnica found 
that 70\% of hybrid and electric vehicle owners had an annual income of \$75,000 or more 
\cite{noauthor_2017_nodate}. The assumption can be made that those who drive electric vehicles can 
afford to pay a toll to use charging lanes, which would benefit highway infrastructure as a whole. 

The people who are directly impacted by the results of our research are electric vehicle owners. 
They are the ones who will see quantifiable benefits in time saved on charging and increased driving range. 
Wireless changing lanes would also benefit poor communities and communities of color. 
The “Drive Change Drive Electric” campaign conducted a survey in which 83\% of respondents cited lack of 
charging stations as a barrier to electric vehicle adoption \cite{nhede_new_2019}. 
A wireless charging lane would encourage drivers to switch to electric vehicles by making charging more 
convenient, decreasing the demand for fossil fuels. Multiple studies have found that the costs of the fossil 
fuel industry “disproportionately fall upon people of color and low-income communities" \cite{noauthor_i-1631_nodate}.
A study published in Ap- plied Geography found that poor communities in Pennsylvania, West Virginia, and Ohio are 
“unequally exposed to pollution from unconventional gas wells" \cite{ogneva-himmelberger_spatial_2015}.
This is one way in which the scope of our research positively impacts a community outside of our intended audience.

Our research can be described as both equity neutral and equity positive. In the areas in which our results may 
have an adverse impact, we have proposed solutions in order to lessen the effects. We have also identified 
communities that we are unintentionally impacting and have determined that the effects of our research are 
beneficial to these communities.