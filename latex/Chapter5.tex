%Chapter 5

\renewcommand{\thechapter}{5}

\chapter{Discussion}

\section{Simulations}
To truly gauge the effectiveness of the simulated model, the results will be applied to two EVs: 
the Nissan Leaf and Tesla Model X. The Nissan Leaf is a budget EV with a 40 kilowatt-hour (kWh) 
battery \cite{noauthor_2021_nodate}. The Tesla Model X is a high-performing EV with a 100 kWh battery 
\cite{noauthor_model_nodate}. There will be two output configurations used from the simulation: the economic 
configuration and maximum output configuration. The economic configuration represents the most optimal 
combination of variables in terms of cost. The maximum output configuration represents the combination 
of variables which output the absolute most charge. See Section \ref{sec: s4.1.3} for the inputs of these 
configurations and how they were determined. 

Calculations and unit conversions are required to show the charge times of the elextric vehicles for each model. 
For the Nissan Leaf, the 40 kwh battery corresponds to 1.44 $\cdot$ 108 J. The Nissan Leaf battery is 360V 
\cite{noauthor_2021_nodate}. Using the formula 
\begin{equation}
    Energy \: [J] = Voltage \: [V] \cdot Coulombs \: [C]
\end{equation}
We determined the battery has 4.0 $\cdot$ 105 C. Using the output data, the distance covered by the vehicle 
is 1600 m (1 mile), and the speed is 30 m/s, so the time the simulated car takes is 53.33 s. In the economic 
configuration, the total charge was 2142.7 C, which corresponds to a charging rate of 40.2 C/s. In the maximum 
output configuration, the total charge is 12334.4 C, which corresponds to a charging rate of 231.3 C/s. 
The time it will take for the maximum output configuration to charge a Nissan Leaf battery is 
\begin{center}
    4.0 $\cdot$ 105 C/231.3 C/s = 1729.5 s = .484 hr
\end{center}
The time it will take for the economic model to charge a Nissan Leaf battery is 
\begin{center}
    4.0 $\cdot$ 105 C/40.2 C/s = 9955.7 s or 2.77 hr
\end{center}
The range of the Nissan Leaf is 149 miles (239792 m), and if the car is going at 30 m/sec, then the battery will 
run out of charge after 7993.07 s or 2.22 hr \cite{noauthor_2021_nodate}.

For the Tesla Model X, the 100 kwh, 350 V battery converts to 3.6 $\cdot$ 108 J \cite{noauthor_model_nodate}. The battery has 
1.0286 $\cdot$ 106 C. The time it will take for the maximum model to charge a Tesla Model X battery is 
\begin{center}
    1.0286 $\cdot$ 106 C/231.3 C/s = 4447.3 s or 1.24 hr
\end{center}
The time it will take for the economic model to charge a Tesla Model X battery is
\begin{center}
    1.0286 $\cdot$ 106 C/40.2 C/s = 25601.1 sec or 7.11 hr
\end{center}
The range of the Tesla Model X is 362 miles (578880 m), and if the car is going at 30 m/sec, then the battery 
will run out of charge after 19296 s or 5.36 hr \cite{noauthor_model_nodate}. See Table \ref{t11} for a summary of charging times.

\begin{table}[H]
    \caption[Charge Time Summary]{Charge Time Summary}
    \begin{center}
    \begin{tabular}{| p{0.5\textwidth} | p{0.2\textwidth} | p{0.2\textwidth} |}
    \hline
    Time [hr] & Nissan Leaf & Tesla Model X \\
    \hline \hline
    Battery life without DWPT & 2.22 & 5.36 \\
    \hline
    Charge with Economic Model & 2.77 & 7.11 \\
    \hline
    Charge with Maximum Output Model & .484 & 1.24 \\
    \hline
    \end{tabular}
    \end{center}
    \label{t11}
\end{table}

Using the data values from Table \ref{t11}, the distance it takes to fully charge the Nissan Leaf and Tesla Model X 
can be found. The distance it will take for the economic model to fully charge a Nissan Leaf is
\begin{center}
    972 sec $\cdot$ 30 m/s = 299160 m or 185.89 miles
\end{center}
The distance it will take for the maximum output model to fully charge a Nissan Leaf is
\begin{center}
    1742.4 sec $\cdot$ 30 m/s = 52272 m or 32.48 miles
\end{center}
The distance it will take for the economic model to fully charge a Tesla Model X is
\begin{center}
    25596 sec $\cdot$ 30 m/s = 767880 m or 477.14 miles
\end{center}
The distance it will take for the  maximum output model to fully charge a Tesla Model X is
\begin{center}
    4464 sec $\cdot$ 30 m/s = 133920 m or 83.21 miles.
\end{center}
See Table \ref{t12} for a summary of the charging distances.

\begin{table}[H]
    \caption[Charge Distance Summary]{Charge Distance Summary}
    \begin{center}
    \begin{tabular}{| p{0.5\textwidth} | p{0.2\textwidth} | p{0.2\textwidth} |}
    \hline
    Distance [miles] & Nissan Leaf & Tesla Model X \\
    \hline \hline
    Battery distance without DWPT & 149 & 360 \\
    \hline
    Charge with Economic Model & 185.89 & 477.14 \\
    \hline
    Charge with Maximum Output Model & 32.48 & 83.21 \\
    \hline
    \end{tabular}
    \end{center}
    \label{t12}
\end{table}

The maximum output model charges both the Nissan Leaf and the Tesla Model X faster than its battery depletes, 
so if this configuration were used, cars could use the chagrining lane only when needed and any other lanes at 
other times.  This could be beneficial because cars don’t need to be taking up space in the lane at all times.  
For the economic model, the charging rate is not enough to allow the vehicle to travel for infinite time. 
For the Nissan Leaf the economic model extends the battery life to about 11.2 hr and for the Tesla Model X it 
extends it to about 21.8 hr. Further, the maximum model can charge a budget EV very quickly. Although this model 
is not cost-efficient, the economic model still charges a budget EV at a fast enough rate to increase the range 
of the EV. Even higher performing EVs like the Tesla Model X will benefit greatly from both the economic and 
maximum models. The economic model provides enough power to extend both a budget and high-end electric vehicle’s 
battery length so that it is long enough for most trips. The new battery life is longer than recommended driving 
time for a single driver, so only trips with multiple drivers remain impacted by the range electric vehicles. 

\section{Experimental Model}
From the experiments, observations are expected to more or less confirm the results from simulation testing. 
Ideally, the relationships discovered from simulations would appear in the experimental model data as well. 
Furthermore, the experimental model allows for relatively quick but extensive data observation, which provides 
insight on how to improve simulations as well. In other words, the experimental model aids with an iterative process 
towards understanding the relationships relevant to wireless power transfer. Because the simulations provide a 
theoretical output for charge, test cases are necessary to determine the efficiency of DWPT in real life. 
Ultimately, the data and values collected from the experimental model would be crucial in proving the feasibility 
of wireless power transfer.

\section{Comparison to Existing Research}
The primary goal of our research has been to contribute to the growing body of knowledge dedicated to improving 
the efficiency and convenience of EVs. The Society of Automotive Engineering (SAE) J2954 standard is one notable 
recent advancement in this field. This standard sets industry-wide specifications for the interoperability of 
stationary wireless power transfer (WPT) systems in light-duty EVs. Our research can contribute to the expansion 
of these standards in the near future. There are two major differences between SAE J2954 and the scope of our research. 
First, SAE J2954 sets standards for stationary WPT while our research focused on dynamic WPT. 
Second, the WPT systems described in SAE J2954 utilize AC power while we tested a system using DC power. 
As mentioned in Section \ref{sec: s4.1}, we used some of the standards from SAE J2954 to define ranges of values of 
the variables tested in our simulations. We found that the values that resulted in the largest possible simulated 
charge output coincided with the standards set by the SAE, which validates our results. 

When running our simulations, we made assumptions that simplified the calculations due to constraints discussed 
in the following section. One of those assumptions was that the coils used in the DWPT system were perfectly 
wound circular coils of a constant radius. The simulations were also run assuming perfect alignment between the 
transmitting and receiving coils. In an analysis conducted by members of the Institute of Electrical and 
Electronics Engineers (IEEE), it was determined that solenoid coils were “smaller, lighter, [and] more tolerant of 
misalignment in a middle and large air gap” than circular coils \cite{fujita_dynamic_2017}. As mentioned briefly in 
Section \ref{sec: s4.1} and in the following section, the simulation and testing of different types of 
coils is one possible continuation of our research.