%Chapter 6

\renewcommand{\thechapter}{6}

\chapter{Future Directions}
\section{Simulations}
With more time and resources, the scope of our research can be expanded to mitigate the limitations highlighted in 
previous sections. Due to time constraints, a limited number of simulations were run that produced graphs showing 
only general trends with some error. The simulations were also run assuming ideal conditions. While this was 
sufficient to answer our research questions, more accurate results are needed if this system is to be physically 
implemented in the future. We first recommend that the numerical simulation be run with a step of .02 m or smaller 
between calculations to reduce error to a negligible value. The next recommendation would be to simulate real world 
conditions. This would include running simulations considering the real permeability of the space between the 
transmitting and receiving coils. For the purposes of our research, the simulation only considered perfect coils 
with a single radius. In the future, the simulation could be adapted to map different types of coils, such as pancake 
coils, to determine which configuration produces the greatest amount of power transfer. The simulation could also be 
adapted to determine how x- and y-components of the magnetic field contribute to the flux, and to determine the true 
efficiency of this power transfer system. A final recommendation for the improvement of the simulation would be to 
simulate imperfect alignment to determine whether that has significant effects on the charge output. If perfect 
alignment is required for maximum power output, another future step could be to incorporate this requirement in 
autonomous function.

\section{Experimental Model}
The purpose of the experimental model built as part of our research was to verify the results of the simulation. 
Due to material and space constraints, the test coils were arranged on a circular track. This does not accurately 
represent the alignment of coils on a straight track (i.e., a highway). The assumption was made that this model 
would scale to real world conditions, although this was not verified. Future endeavors could include improving the 
experimental model by creating a scale model of a standard highway and using materials that more accurately reflect 
real world conditions (such as concrete). The experimental model could also be improved by allowing for different 
types of coils to be tested, much like what was recommended for the simulation. The scope of the experimental model 
could also be expanded in the future to determine the effects of weather.